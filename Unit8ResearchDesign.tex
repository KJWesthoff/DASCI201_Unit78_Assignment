%%%%%%%%%%%%%%%%%%%%%%%%%%%%%%%%%%%%%%%%%
% Journal Article
% LaTeX Template
% Version 2.0 (February 7, 2023)
%
% This template originates from:
% https://www.LaTeXTemplates.com
%
% Author:
% Vel (vel@latextemplates.com)
%
% License:
% CC BY-NC-SA 4.0 (https://creativecommons.org/licenses/by-nc-sa/4.0/)
%
% NOTE: The bibliography needs to be compiled using the biber engine.
%
%%%%%%%%%%%%%%%%%%%%%%%%%%%%%%%%%%%%%%%%%

%----------------------------------------------------------------------------------------
%	PACKAGES AND OTHER DOCUMENT CONFIGURATIONS
%----------------------------------------------------------------------------------------

\documentclass[
	letterpaper, % Paper size, use either a4paper or letterpaper
	12pt, % Default font size, can also use 11pt or 12pt, although this is not recommended
	unnumberedsections, % Comment to enable section numbering
	twoside, % Two side traditional mode where headers and footers change between odd and even pages, comment this option to make them fixed
]{LTJournalArticle}

\addbibresource{bibliography.bib} % BibLaTeX bibliography file

\runninghead{Research Design Review} % A shortened article title to appear in the running head, leave this command empty for no running head

\footertext{\textit{Unit 08 Assignment} (MICS/DATSCI 201, Spring 2025)} % Text to appear in the footer, leave this command empty for no footer text

\setcounter{page}{1} % The page number of the first page, set this to a higher number if the article is to be part of an issue or larger work

%----------------------------------------------------------------------------------------
%	TITLE SECTION
%----------------------------------------------------------------------------------------

\usepackage[title,toc,titletoc]{appendix}
\usepackage{titlesec}
\usepackage{lscape}
\usepackage{fontawesome}


\title{Research Design Review} % Article title, use manual lines breaks (\\) to beautify the layout}

% Authors are listed in a comma-separated list with superscript numbers indicating affiliations
% \thanks{} is used for any text that should be placed in a footnote on the first page, such as the corresponding author's email, journal acceptance dates, a copyright/license notice, keywords, etc
\author{
	Karl-Johan Westhoff \\
	email \href{mailto:kjwesthoff@berkeley.edu}{kjwesthoff@berkeley.edu}
}

% Affiliations are output in the \date{} command
\date{UC Berkeley School of Information \\
MIDS Course 201 Spring 2025, Section 6 (Brooks Ambrose)
}

% % Full-width abstract
% \renewcommand{\maketitlehookd}{%
% 	\begin{abstract}
% 		\noindent Lorem ipsum dolor sit amet,rta porttitor.
% 	\end{abstract}
% }

%----------------------------------------------------------------------------------------
\setcounter{tocdepth}{5}
\setcounter{secnumdepth}{5}
\usepackage[title]{appendix}

\begin{document}
\onecolumn
\maketitle % Output the title section



%----------------------------------------------------------------------------------------
%	ARTICLE CONTENTS
%----------------------------------------------------------------------------------------


\section*{Internal Memo, Content Moderation}

Looking at content moderation beyond our own company, there are many organizations with similar challenges to ours. The purpose of content moderation is to ensure that the content complies with community guidelines. The guidelines can be grouped into sections based on the rules specified by each part:

\begin{enumerate}[label=(\Alph*)]
	\item Societal rules: Laws, standards and regulations
	\item Malicious actors: Spam, unwanted commercial and political soliciting, forum hijacking, malware distribution
	\item Social rules: Inappropriate content, Hate speech, Racism, Homophobia
	\item Rules for our company: How do we want people to engage with each other, the debate 'tone', and handling influence, for example from competitors.
\end{enumerate}

Each part can be addressed individually and processed in parallel, and there are various solutions for handling content moderation both in-sourced and out-sourced, AWS for example offers a pay-per-post solution\cite{AWS_contentModeration}. The aim of this memo is to review the technologies that are available for content moderation and how we should measure how well it works.


\section{Potential solutions}

\subsection{Automation, Machine Learning and AI}
An overview of Machine Learning (ML) for content moderation was done by Chowdhury in \cite{StanfordPrimer}. Two automated techniques are used in the industry:
\begin{itemize}
	\item Matching models: Where content is compared (with various degree of similarity) against a database of known unwanted content. This is best for things that are clearly identifiable such as features present in pictures, specific words and language, plagiarism etc.
	\item Predicting models: Predicts the likelihood of a text or image violating a set of rules defined by the categories above. For example analyses a text to identify if the content violates a law.
\end{itemize}

Matching models are effective for raw filtering against well defined rules and especially applicable to (A) and (B) above. Predicting models provide a probability that 'something' is in violation of a rule and includes context rather than just looking at a 'bag of words'.

\subsubsection*{Features of ML and AI based solutions}
\begin{itemize}
	\item Can scale quickly as traffic increases and decreases on our forums
	\item Machine Learning based content moderation has historically been prone to over filtering (false positives), with a hands-off approach, operators have erred towards "better catch too many than too few". However, this is a developing field and performance is expected to increase
	\item Using more advanced methods beyond ML such as Large Language Models (LLM's) on every forum post and thread is currently too expensive. However, the  cost is expected to decrease
\end{itemize}

\subsection{Human Moderators}
Content moderation using people trained to precisely evaluate the content.
\subsubsection*{Features, Manual Moderation}
\begin{itemize}
	\item Flexible, high quality evaluation capturing nuances
	\item Immediate reaction to and prediction of new trends in society
	\item Low initial cost, high operating cost (requires qualified manpower)
	\item Scaling with forum traffic is slow, cost increases with increased traffic
	\item Large cost impact in case of the forum being 'attacked' with a flood of malicious posts
\end{itemize}



\subsection{Hybrid Approach}
The hybrid approach seems to be the industry norm, with efforts to increase the automated part. Sircar et al. looked at the company 'X''s transparency report in \cite{ForbesXFindings}. The largest players in social media are trying to increase the reliance on AI, with some unexpected side effects: AI seems not yet to be ready to do the work alone. "AI systems struggle to identify the full spectrum of harmful behaviors"\cite{ForbesXFindings}, training data are skewed to 'western' societies and misses hate speech from other cultures and does not understand features such as sarcasm and 'secret language'\footnote{Like kids inventing their own language and expressions}.

\subsubsection*{Features, Hybrid Moderation}
\begin{itemize}
	\item Cost-efficient compared to a purely human approach
	\item Can react quickly to new language, words and trends
	\item Lets AI/ML do the parts it is good at, start with from 'Matching Models' and expand from there
	\item More complex to implement and requires constant adjustment as automated processes can do more
	\item Operators can transition from doing the content moderation itself to using domain knowledge for improving the automation
\end{itemize}

\section{Evaluation}

\subsection{Trial Stage Evaluation}
The purpose is to apply measurements of how good the proposed content moderation approaches above work prior to committing fully to a solution. In this case we do have a set of labeled data from our current content moderation which can serve as training and test data for evaluation.

\subsubsection*{Trial Test Setup}
\begin{itemize}
	\item Run each solution with the test data
	\item Evaluate each solution, especially for false negatives and false positives
	\item Compare and review the solutions including cost of implementation and operation
	      \begin{itemize}
		      \item The possible future improvements to automated solutions should be evaluated and included
		      \item The evaluation should be done as a pilot project as part of the research design
	      \end{itemize}
\end{itemize}

\subsubsection*{Trial Metrics}
Classical machine learning metrics\cite{ClarenceChio}ch.2:
\begin{itemize}
	\item Precision, How good did we predict, false positives
	\item Recall, How good did we predict false negatives
	\item F1 Score, combination of precision and recall
\end{itemize}

\subsection{Evaluation Post-Deployment}
We can use the same Machine Learning metrics as for the trials by sampling data for bench marking as we go along. However, the criteria for what we consider inappropriate may change over time\footnote{especially for (D) above regarding company defined rules}. This can be evaluated using surveys of customer satisfaction and usages statistics. Cost vs. efficiency of our content moderation should be evaluated periodically as new technology evolves.

\subsubsection*{In-Operation Metrics}
\begin{itemize}
	\item Periodical sampling and evaluation of ML
	      \begin{itemize}
		      \item Precision
		      \item Recall
		      \item F1 Score
	      \end{itemize}
	\item Customer satisfaction surveys, will indirectly compare to the rest of the industry
	\item Usage statistics, tied into business goals and performance metrics
	\item Periodic evaluation of cost
\end{itemize}




\section*{Conclusion}
Three solutions for content moderation were proposed and methods for evaluation during design and after implementation were evaluated. The solutions represent the current industry standards and near future improvements. The evaluation metrics are based on classical methods and comparison to industry standards and market competitors.


%----------------------------------------------------------------------------------------
%	 REFERENCES
%----------------------------------------------------------------------------------------
\printbibliography % Output the bibliography
%----------------------------------------------------------------------------------------

\subsubsection*{Notes on this assignment:}
Word count of the body text is approx. 970 words, A three page 'lorem ipsum' document with 5 paragraphs and 1in margins would be up to 950 words, I thereby argue that the assignment fulfills the assignment length requirement by being equivalent to a Three page document. \\
I fed the assignment to ChatGPT, it gave me some ideas to the overall structure of the assignment, and the 3 types of content moderation (although it was somewhat obvious) you can find my 'conversation' here: \url{https://chatgpt.com/share/67bcb191-e528-8008-b806-3b2f8c463a0f}
%----------------------------------------------------------------------------------------
%	 Appendices
%---------------------------------------------------------------------------------------


%\begin{appendices}
%	\onecolumn
%
%	\section{Appendix}
%
%
\end{document}
